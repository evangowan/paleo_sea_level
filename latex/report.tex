\documentclass[a4paper,12pt]{article}

\usepackage[margin=2cm]{geometry}
\usepackage{graphicx}
\usepackage{tabularx}
\usepackage{supertabular}
\usepackage{float}
\usepackage{multirow}
\usepackage{import}
\usepackage[sort]{natbib}
\usepackage{times}
\usepackage{url}
\def\UrlBreaks{\do\/\do-\do\_}
\usepackage[hidelinks]{hyperref}
\usepackage{placeins}
\usepackage[nobottomtitles]{titlesec}

\usepackage[nottoc,notlot,notlof]{tocbibind}

\usepackage{fontspec}
\setmainfont{FreeSerif}
\setsansfont{FreeSans}
\setmonofont{FreeMono}

\usepackage{polyglossia}
\setdefaultlanguage{english}
\setotherlanguages{russian}

% Japanese text
\usepackage{zxjatype}
\setjamainfont{ipaexm.ttf}

\setlength{\parindent}{0pt}
\setlength{\parskip}{1em}

\begin{document}


\title{Comparision of calculated and measured paleo-sea level proxies with PaleoMIST 1.0, Report 1, version 2}

\author{Evan James Gowan}
    
\date{}

\maketitle

\normalsize As a supplement to ``\emph{A new global ice sheet reconstruction for the past 80\,000 years}" by Evan J. Gowan, Xu Zhang, Sara Khosravi, Alessio Rovere, Paolo Stocchi, Anna L. C. Hughes, Richard Gyllencreutz, Jan Mangerud, John-Inge Svendsen \& Gerrit Lohmann

\textbf{Report 1}: Comparing six lower mantle models using the maximal MIS 3 scenario of PaleoMIST 1.0.

\newpage

\tableofcontents

\newpage

\section{Purpose of this document}

In this report there is a detailed summary, including plots, of a worldwide compilation of paleo-sea level data, and six ice sheet-Earth models. In this particular report, we compare the standard version of PaleoMIST 1.0 (with 2500 year time steps and using a lower mantle viscosity of $4\times10^{22}$ Pa~s), with five other Earth models with viscosity values ranging between $10^{21}$ and $10^{23}$. When developing PaleoMIST 1.0, a variety of lower mantle viscosity values were tested, and it was found that a value approaching $10^{23}$ Pa~s provided the best trade-off between increasing the amount of ice in the center of the Laurentide Ice Sheet and fitting the sea level data. This ended up being true for the Eurasian ice sheets as well. PaleoMIST 1.0 was tuned to an Earth model with a viscosity of $4\times10^{22}$ Pa~s, but the comparison shown in this document demonstrate that a slightly higher value of $10^{23}$ Pa~s provides an even better fit.

The accompanying paper is \citet{GowanEtal2021b}.

Note that earlier versions of this report used the minimal scenario of PaleoMIST 1.0, but I now prefer the maximal scenario. The load is now linearly interpolated to 500 year time steps, which should provide a more realistic representation of ice sheet evolution, and therefore reduce the loading effects in the time series.


\import{./}{changelog.tex}

\section{Summary of ice and Earth models}


The main models included here are from PaleoMIST. This is a global ice sheet reconstruction at a very crude 2500 year time step. I have started to use a 500 year interpolated version, which should produce more accurate results in ice covered areas, though it makes less impact in far field regions.

For this document, I use PaleoMIST 1.0. The minimal MIS~3 configuration reconstruction is PM\_1, while the maximal configuration is PM\_1\_A

For the Earth models, I created a shorthand scheme during my PHD, which I have continued to use. A full explanation can be found on the github page:

\url{https://github.com/evangowan/icesheet/blob/master/global/earth_model_format_codes.txt}

The full description of each model compared in this document is in this section.



\subsection{Ice models}

\import{temp/}{ice_models.tex}

\subsection{Earth models}

\import{temp/}{earth_models.tex}

\newpage


\import{./}{compilations.tex}

\newpage

\import{./}{statistics.tex}


\newpage

\import{figure_tex/}{summary.tex}

\clearpage

\newpage

% bibliography
\bibliographystyle{copernicus}
\bibliography{references.bib}

\end{document}
