\documentclass[a4paper,12pt]{article}

\usepackage[margin=2cm]{geometry}
\usepackage{graphicx}
\usepackage{tabularx}
\usepackage{supertabular}
\usepackage{float}
\usepackage{multirow}
\usepackage{import}
\usepackage[sort]{natbib}
\usepackage{times}
\usepackage[hidelinks]{hyperref}


\usepackage{fontspec}
\setmainfont{FreeSerif}
\setsansfont{FreeSans}
\setmonofont{FreeMono}

\usepackage{polyglossia}
\setdefaultlanguage{english}
\setotherlanguages{russian}


\setlength{\parindent}{0pt}
\setlength{\parskip}{1em}

\begin{document}


\title{Comparision of calculated and measured paleo-sea level}

\author{Evan James Gowan \emph{et al}}
    
\date{}

\maketitle

\tableofcontents

\newpage

\section{Summary of ice and Earth models}


In order to make the figures compact, I have made shorthand codes for the ice and Earth models. I calculate each ice sheet separately, and the numbers refer to the ``run number", which is a sequential number that I used to distinguish git commits (see \url{https://github.com/evangowan/icesheet}). The ice model numbering scheme is as follows:

``North America"\_``Europe"\_``Antarctica"\_``Patagonia"

For the Earth models, I created a shorthand scheme during my PHD, which I have continued to use. A full explanation can be found on the github page:

\url{https://github.com/evangowan/icesheet/blob/prelim/global/earth_model_format_codes.txt} % change when final version is pushed

The full description of each model compared in this document is in this section.



\subsection{Ice models}

\import{temp/}{ice_models.tex}

\subsection{Earth models}

\import{temp/}{earth_models.tex}

\newpage


\import{./}{compilations.tex}

\newpage

\import{./}{statistics.tex}


\newpage

\import{figure_tex/}{summary.tex}

\clearpage

\newpage

% bibliography
\bibliographystyle{copernicus}
\bibliography{references.bib}

\end{document}
